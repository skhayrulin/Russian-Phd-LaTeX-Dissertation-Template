Новые методы, разрабатываемые математиками, физиками и программистами в сотрудничестве с биологами, позволяют создавать компьютерные модели - в том числе сложных живых систем, которые могут адекватно воспроизводить их свойства. Растущий интерес ученных связан со стремлением приблизится к пониманию биологических и физических принципов функционирования живых организмов и их подсистем. В качестве примера можно привести как исследования функций живого нейрона, так и целых нервных систем живых организмов с использованием теории математического и компьютерного моделирования. Подробные математические модели, учитывающие базовые принципы распространения, обработки и хранения информации в нейронных контурах чрезвычайно востребованы для дальнейших исследований не только в области биологии и медицины, но и в области создания искусственного интеллекта, как копии естественного.

В то же время, совершенно очевидно, что без развития таких направлений, как суперкомпьютерные вычислительные системы, модели и методы создания программ и программных систем для параллельной и распределенной обработки данных, языков и инструментальных средств параллельного программирования, появление более абстрактных методов описания математических моделей сложных систем вряд ли было бы возможно. Зачастую разработка компьютерных моделей биологических объектов связана с анализом больших объемов данных; в тоже время, чем точнее модель, тем больше физических и физиологических особенностей необходимо учитывать, что приводит к необходимости быстрой и эффектинвой обработки больших массивов данных. Это в свою очередь зависит от развития методов и программных протоколов межпроцессного взаимодействия как на сетевом уровне, так и на уровне операционной системы.

{\actuality} Модель сложной системы проще представлять как комбинацию или ансамбль нескольких моделей, взаимодействующих друг с другом в соответствии с некоторыми протоколами. При этом методы и подходы к реализация каждой независимой компоненты подобной комплексной системы могут в значительной мере отличаться. Если подойти к проблеме создания модели контура нейросети, то нужно учитывать, что она не может существовать в вакууме, то есть нужно уметь моделировать корректный поток данных поступающих на вход, а также уметь анализировать выходящий поток данных, для определения адекватности модели.

Например, в рамках проекта OpenWorm, целью которого является полномасштабное моделирование нематоды  С. elegans, различные системы живого организма моделируются отдельно, и при этом осуществляется обмен данными между ними в процессе симуляции. Так, модель нервной системы описывается в декларативной  форме на специализированном языке  NeuroML (расширение XML), которая затем интерпретируется симулятором  NEURON \cite{Carnevale2006}. Мышечная система нематоды и ее гидростатический скелет представлены как модель, описанная численным методом  PCI SPH - predictor corrector smoothed particle hydrodynamics (модификация метода SPH). Взаимодействие между моделями организуется с помощью низкоуровневого API в реальном времени. Так же различные модификации метода SPH используются при моделировании, кровеносных систем \cite{Caballero2017} и конечно в других областях науки и техники как, например, машинной графики  или компьютерной анимации \cite{Solenthaler2013} (Mass Preserving Multi-Scale SPHPixar Technical Memo \#13-04). Несмотря на большую популярность и гибкость, которую предоставляет метод, в отличии от методов конечных элементов, он обладает таким значительным недостатком, как низкая производительность. В представленной работе  предлагается ряд алгоритмов, с помощью которых предполагается увеличить производительность численных методов SPH для задач связанных с моделированием гидродинамики и механики биологических систем и процессов.

Решение вышеперечисленных и многих других проблем лежит в применении объектно-ориентированного подхода и современных методов рационального планирования процессов. Ввиду того, что решение почти любой задачи в современном мире можно представить в виде компьютерной программы, объектно-ориентированные подходы в программировании особенно важны.

% \ifsynopsis
%   Этот абзац появляется только в~автореферате.
%   Для формирования блоков, которые будут обрабатываться только в~автореферате,
%   заведена проверка условия \verb!\!\verb!ifsynopsis!.
%   Значение условия задаётся в~основном файле документа (\verb!synopsis.tex! для
%   автореферата).
% \else
%   Этот абзац появляется только в~диссертации.
%   Через проверку условия \verb!\!\verb!ifsynopsis!, задаваемого в~основном файле
%   документа (\verb!dissertation.tex! для диссертации), можно сделать новую
%   команду, обеспечивающую появление цитаты в~диссертации, но~не~в~автореферате.
% \fi

% {\progress}
% Этот раздел должен быть отдельным структурным элементом по
% ГОСТ, но он, как правило, включается в описание актуальности
% темы. Нужен он отдельным структурынм элемементом или нет ---
% смотрите другие диссертации вашего совета, скорее всего не нужен.

{\aim} разработка алгоритмов и программных технологий,  повышающих эффективность процессов обработки данных в вычислительных машинах и комплексах для семейства алгоритмов моделирования динамики несжимаемой жидкости PCI SPH - за счет использования в параллельном режиме  всех доступных вычислительных узлов,  интеграции разработанных методов в единую программную систему, \colorbox{red!30}{ а также реализации программных средств и человеко-машинных интерфейсов для мультимедийного взаимодействия с симулятором NEURON}.

{\methods}Для решения поставленных задач нами использовались методы математического моделирования, методы массивно-параллельной обработки данных, методы объектно-ориентированного анализа и проектирования систем. Параллельная модель вычислений реализуется на базе многопоточности. В исследованиях применялись методы системного, объектно-ориенти­рованного и параллельного программирования, имитационного моделирования и теории структур данных. Разработка программного обеспечения проводилась на языке C++ и Python, с использованием технологий OpenCL, OpenGL.

  {\novelty} Проведенные исследования позволили разработать и предложить ряд новых определений, формальных моделей и алгоритмов, которые могут быть применены при разработки алгоритмов и программ в таких областях как биомеханика, гидродинамика и механика сплошных сред.

Предложен эффективный алгоритм распределения данных между вычислительными узлами, который позволяет увеличить производительность алгоритма PCI SPH благодаря использованию всех доступных вычислительных устройств.

\begin{itemize}
  \item{Алгоритм динамически синхронизирует распределение данных между узлами, что позволяет сохранять постоянную  оптимальную нагрузку.}
  \item{Предложена модификация параллельного алгоритма цифровой сортировки для массивов комплексных структур данных.}
  \item{Данные распределяются в соответствии с вычислительными мощностями устройства, что зависит от конкретных характеристик устройства. }
  \item{Предложен и реализован новый алгоритм поиска соседей для Лагранжевых методов моделирования механики сплошных сред. Алгоритм гарантирует выбор наиболее близких соседних частиц.}
\end{itemize}

{\defpositions}
\begin{enumerate}
  \item Разработан и обоснован новый алгоритм поиска соседей для класса методов моделирования механики сплошных сред SPH
  \item Разработан и обоснован новый алгоритм распределенной обработки данных для метода моделирования динамики жидкости PCI SPH
  \item Полученные расчетные данные позволяют утверждать, что предлагаемое решение дает прирост производительности для больших конфигураций практически в два раза (в зависимости от вычислительного кластера)
  \item Разработан и обоснован новый алгоритм синхронизации данных между вычислительными узлами, вычисление весовых коэффициентов при этом вычисляется автоматически на основе предложенной эвристической функции
\end{enumerate}

Для~достижения поставленной цели необходимо было решить следующие {\tasks}:
\begin{enumerate}
  \item Разработать алгоритм распределения данных по нескольким вычислительным устройствам для класса методов PCI SPH.
  \item Создать алгоритм динамической синхронизации вычислений и данных между вычислительными устройствами.
  \item Предложить параллельный  алгоритм эффективного поиска соседей для класса методов PCI SPH.
  \item Разработать модель данных и формальную модель  вычислений, для вышеперечисленных алгоритмов.
  \item Реализовать комплекс алгоритмов и проблемно ориентированных программ для проведения вычислительного эксперимента по оценке роста производительности для алгоритмов PCI SPH.
  \item Разработать алгоритм параллельной сортировки массивов комплексных структур данных
\end{enumerate}


{\influence} В результате проведенных исследований созданы ряд алгоритмов и программных систем, которые могут используется для моделирования механики сплошных сред. В том числе для моделирования биологических систем. Например, гидростатического скелета нематоды C. elegans, мышечной системы и окружения. Другим примером может быть использование полученных алгоритмов при моделировании капиллярной системы кровеносных сосудов. Область применения не ограничена лишь биологическими объектами − данные алгоритмы могут быть полезны при моделировании гидродинамики в принципе. Также разработаны новые методы, позволяющие визуализировать сложные структуры  биологических нейросетей, которые упрощают взаимодействие \fixme{Тут еще не все явно будет ли это в диссертации с симулятором NEURON}.

{\reliability} и обоснованность полученных результатов обеспечивается адекватными постановками задач и применяемыми математическими моделями, использованием современных методов разработки программ на основе объектно­ ориентированного и проце­дурного программирования, а также подтверждается результатами тестовых расчётов в сопоставлении с аналитическими оценками.

\ifnumequal{\value{bibliosel}}{0}
{%%% Встроенная реализация с загрузкой файла через движок bibtex8. (При желании, внутри можно использовать обычные ссылки, наподобие `\cite{vakbib1,vakbib2}`).
  {\publications} Основные результаты по теме диссертации изложены
  в~XX~печатных изданиях,
  X из которых изданы в журналах, рекомендованных ВАК,
  X "--- в тезисах докладов.
}%
{%%% Реализация пакетом biblatex через движок biber
  \begin{refsection}[bl-author]
    % Это refsection=1.
    % Процитированные здесь работы:
    %  * подсчитываются, для автоматического составления фразы "Основные результаты ..."
    %  * попадают в авторскую библиографию, при usefootcite==0 и стиле `\insertbiblioauthor` или `\insertbiblioauthorgrouped`
    %  * нумеруются там в зависимости от порядка команд `\printbibliography` в этом разделе.
    %  * при использовании `\insertbiblioauthorgrouped`, порядок команд `\printbibliography` в нём должен быть тем же (см. biblio/biblatex.tex)
    %
    % Невидимый библиографический список для подсчёта количества публикаций:
    \printbibliography[heading=nobibheading, section=1, env=countauthorvak,          keyword=biblioauthorvak]%
    \printbibliography[heading=nobibheading, section=1, env=countauthorwos,          keyword=biblioauthorwos]%
    \printbibliography[heading=nobibheading, section=1, env=countauthorscopus,       keyword=biblioauthorscopus]%
    \printbibliography[heading=nobibheading, section=1, env=countauthorconf,         keyword=biblioauthorconf]%
    \printbibliography[heading=nobibheading, section=1, env=countauthorother,        keyword=biblioauthorother]%
    \printbibliography[heading=nobibheading, section=1, env=countauthor,             keyword=biblioauthor]%
    \printbibliography[heading=nobibheading, section=1, env=countauthorvakscopuswos, filter=vakscopuswos]%
    \printbibliography[heading=nobibheading, section=1, env=countauthorscopuswos,    filter=scopuswos]%
    %
    \nocite{*}%
    %
    {\publications} Основные результаты по теме диссертации изложены в~\arabic{citeauthor}~печатных изданиях,
    \arabic{citeauthorvak} из которых изданы в журналах, рекомендованных ВАК\sloppy%
    \ifnum \value{citeauthorscopuswos}>0%
      , \arabic{citeauthorscopuswos} "--- в~периодических научных журналах, индексируемых Web of~Science и Scopus\sloppy%
    \fi%
    \ifnum \value{citeauthorconf}>0%
      , \arabic{citeauthorconf} "--- в~тезисах докладов.
    \else%
      .
    \fi
  \end{refsection}%
  \begin{refsection}[bl-author]
    % Это refsection=2.
    % Процитированные здесь работы:
    %  * попадают в авторскую библиографию, при usefootcite==0 и стиле `\insertbiblioauthorimportant`.
    %  * ни на что не влияют в противном случае
    \nocite{vakbib2}%vak
    \nocite{bib1}%other
    \nocite{confbib1}%conf
  \end{refsection}%
  %
  % Всё, что вне этих двух refsection, это refsection=0,
  %  * для диссертации - это нормальные ссылки, попадающие в обычную библиографию
  %  * для автореферата:
  %     * при usefootcite==0, ссылка корректно сработает только для источника из `external.bib`. Для своих работ --- напечатает "[0]" (и даже Warning не вылезет).
  %     * при usefootcite==1, ссылка сработает нормально. В авторской библиографии будут только процитированные в refsection=0 работы.
  %
  % Невидимый библиографический список для подсчёта количества внешних публикаций
  % Используется, чтобы убрать приставку "А" у работ автора, если в автореферате нет
  % цитирований внешних источников.
  % Замедляет компиляцию
  \ifsynopsis
    \ifnumequal{\value{draft}}{0}{
      \printbibliography[heading=nobibheading, section=0, env=countexternal, keyword=biblioexternal]%
    }{}
  \fi
}

При использовании пакета \verb!biblatex! будут подсчитаны все работы, добавленные
в файл \verb!biblio/author.bib!. Для правильного подсчёта работ в~различных
системах цитирования требуется использовать поля:
\begin{itemize}
  \item \texttt{authorvak} если публикация индексирована ВАК,
  \item \texttt{authorscopus} если публикация индексирована Scopus,
  \item \texttt{authorwos} если публикация индексирована Web of Science,
  \item \texttt{authorconf} для докладов конференций,
  \item \texttt{authorother} для других публикаций.
\end{itemize}
Для подсчёта используются счётчики:
\begin{itemize}
  \item \texttt{citeauthorvak} для работ, индексируемых ВАК,
  \item \texttt{citeauthorscopus} для работ, индексируемых Scopus,
  \item \texttt{citeauthorwos} для работ, индексируемых Web of Science,
  \item \texttt{citeauthorvakscopuswos} для работ, индексируемых одной из трёх баз,
  \item \texttt{citeauthorscopuswos} для работ, индексируемых Scopus или Web of~Science,
  \item \texttt{citeauthorconf} для докладов на конференциях,
  \item \texttt{citeauthorother} для остальных работ,
  \item \texttt{citeauthor} для суммарного количества работ.
\end{itemize}
% Счётчик \texttt{citeexternal} используется для подсчёта процитированных публикаций.

Для добавления в список публикаций автора работ, которые не были процитированы в
автореферате требуется их~перечислить с использованием команды \verb!\nocite! в
\verb!Synopsis/content.tex!.

{\probation}
Результаты работы были представлены на международных научных конференциях, включая Mathematical Modeling and High--Performance Computing in Bioinformatics, Biomedicine and Biotechnology (MM \& HPC-2014, MM \& HPC-2016), Новосибирск, Россия (2014, 2016), PSI 2015 (10-th Ershov Informatics Conference) Новосибирск, Россия (2015), 24th Annual Computational Neuroscience Meeting: CNS 2015, Прага, Чешская Республика, 22-th Annual Computational Neuroscience Meeting(OCNS 2013) в г. Париж, Франция; 5-th Proc. Neuroinformatics (INCF 2012) в г. Мюнхен, Германия; 4--th Proc. Neuroinformatics (INCF 2011) в г. Бостон, США; Современные проблемы математики, информатики и биоинформатики (2011) в г. Новосибирск, Россия; XLIX Международная научная студенческая конференция
«Студент и научно-технический прогресс» 2010-2011 гг. в г. Новосибирск, Россия; Региональный этап международного конкурса Microsoft Imagine Cup в 2010 г. в городе Томск, Россия, работа заняла 3 место и была отмечена грамотой.Работа была представлена на рабочем семинаре «Наукоемкое программное обеспечение» конференции памяти академика А. П. Ершова «Перспективы систем информатики»,  а также на ряде семинаров и встреч с отечественными и зарубежными коллегами.

{\contribution} Автор принимал активное участие в реализации описанных выше алгоритмов которая является достаточно трудоемкой задачей. Наибольший вклад в первой задаче автор диссертации внес в разработку и реализацию следующих алгоритмов: распределенной обработки данных при использовании нескольких вычислительных узлов, поиска соседей и синхронизации данных, алгоритма параллельной сортировки массивов специальным образом структурированных данных, создании и тестировании алгоритмов и программ, проведении расчетов и интерпретации результатов.


